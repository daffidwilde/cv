\documentclass[10.5pt,a4paper,sans]{moderncv}

\nopagenumbers%
\moderncvstyle{casual}
\moderncvcolor{nord}

\usepackage[T1]{fontenc}
\usepackage{roboto}
\usepackage[margin=2cm, includefoot, footskip=1em]{geometry}
\recomputelengths

\definecolor{myurl}{HTML}{5e81ac}

% URLs
\AfterPreamble{\hypersetup{
    colorlinks=true,
    urlcolor=myurl,
}}

\newcommand{\arxiv}[1]{%
    \href{https://arxiv.org/abs/#1}{\nolinkurl{arXiv:#1}}%
}

\newcommand{\doi}[1]{%
    \href{https://doi.org/#1}{\nolinkurl{doi:#1}}%
}

% Bibliographies
\usepackage[backend=bibtex, style=alphabetic]{biblatex}
\addbibresource{bibliography.bib}

\newlength{\bibbeforelabelsep}
\setlength{\bibbeforelabelsep}{3\labelsep}
\setlength{\biblabelsep}{2\labelsep}

\defbibenvironment{bibliography}
  {\list
     {\printtext[labelalphawidth]{%
        \printfield{labelprefix}%
        \printfield{labelalpha}%
        \printfield{extraalpha}}}
     {\setlength{\labelwidth}{\labelalphawidth}%
      \setlength{\leftmargin}{\labelwidth}%
      \setlength{\labelsep}{\biblabelsep}%
      \addtolength{\leftmargin}{\labelsep}%
      \addtolength{\leftmargin}{\bibbeforelabelsep}%
      \setlength{\itemsep}{\bibitemsep}%
      \setlength{\parsep}{\bibparsep}}%
      \renewcommand*{\makelabel}[1]{\hss##1}}
  {\endlist}
  {\item}

\name{Henry}{Wilde\vspace{1pt}}
\email{henrydavidwilde@gmail.com}
\social[twitter]{daffidwilde}
\social[github]{daffidwilde}


\begin{document}
\vspace{-2em}\makecvtitle%

\vspace{-3em}%
\section{PhD thesis}
\cvitem{Title}{%
    \textbf{
        New methods for algorithm evaluation and cluster initialisation with
        applications to healthcare%
    }
}
\cvitem{Supervisors}{\small%%
    Dr Jonathan Gillard, Dr Vincent Knight, Mr Kendal Smith (NHS Wales)
}
\cvitem{Description}{\small%
    This thesis offers a thorough utilisation of machine learning techniques to
    understand variation in healthcare data; this is achieved by assessing,
    creating and applying machine learning techniques in a novel manner. The
    results of this are new perspectives on algorithm evaluation and fair
    clustering, as well as attaining actionable insights into a healthcare
    population using routinely gathered, administrative datasets. In addition to
    these contributions, this thesis is accompanied by a collection of
    well-developed research software packages.
}

\section{Education}

\cventry{2017--2021}%
    {PhD Applied Statistics, OR and Data Analytics}%
    {Cardiff University}{}{}{%
    \begin{itemize}
        \item \emph{Research interests:} healthcare modelling, clustering,
            algorithm evaluation, game theory, research software development.
        \item \emph{Key skills:} confidence in the principles of data science
            and mathematical programming, ability to develop industrial
            relationships, proficiency in \LaTeX\ and Python.
    \end{itemize}
}

\cventry{2014--2017}%
    {BSc Mathematics (First Class Honours)}{Cardiff University}{}{}{%
        \begin{itemize}
            \item \emph{Key areas of study:} methods for data mining, game
                theory, algorithms and heuristics.
            \item \emph{Projects included:} building a simulation of a hospital
                emergency department, and a principal analysis of two
                game-theoretic strategies within an Iterated Prisoner's Dilemma
                tournament.
        \end{itemize}
}

\section{Relevant experience}

\cventry{Feb 2021--present}{Research associate}{Cardiff University}{}{}{%
    I am a member of the mathematical modelling team investigating the potential
    for wastewater sampling in disease prediction, and particularly COVID-19.
    This role is split between two major projects: the WeWASH project with Welsh
    Government, and an international project with the University of Campinas,
    funded by the Global Challenges Research Fund. My main responsibilities are
    in developing, assessing and communicating intuitive statistical models for
    predicting prevalence using an array of data sources, including geospatial
    public health data and chemical analyses.
}

\cventry{2019--present}{Final year project allocation}{Cardiff University}{}{}{%
    I have installed a new framework for allocating dissertations to final year
    students in the School of Biosciences at Cardiff University. The
    software-based framework has reduced the bulk of their workflow down to a
    matter of seconds. In doing so, the allocation is both mathematically fair
    and student-optimal, with almost all students being allocated to their first
    or second choice of project. 
}

\cventry{2018--2019}{MMORS dissertation co-supervisor}{Cardiff University}{}{}{%
    I assisted in the supervision of a MMORS final year project with Dr Vince
    Knight conducting an empirical study of Folk Theorems in repeated games. My
    primary role was to consult on how best to develop the supporting research
    software, and in the writing process.
}

\cventry{2017--2021}{PhD studentship teaching}{Cardiff University}{}{}{%
    Throughout my time as a postgraduate student, I supported a number of
    modules and services as a tutor and assessment advisor. These roles provided
    me with countless opportunities to reinforce teaching and learning in a
    variety of environments, including group, one-on-one, structured and drop-in
    settings. This range has allowed me to demonstrate my nature as a
    mathematician and educator, one who is proactive, analytical, logical, and
    enthusiastic.
}

\clearpage%
\nocite{*}

\printbibliography[nottype=software, title={Publications and pre-prints}]
\printbibliography[type=software, title={Software projects}]

\section{Additional activities}

\cventry{Jan 2020}%
    {SIAM UKIE Annual Meeting 2020}{The University of Edinburgh}{}{}{%
        Received a travel award to present a poster on evolutionary dataset
        optimisation --- including a case study comparing \(k\)-means and DBSCAN
        clustering.
    }

\cventry{May 2019}{Welsh Mathematics Colloquium}{Gregynog Hall}{Powys}{}{%
    Gave an in-depth talk on the mathematical principles of evolutionary dataset
    optimisation and some of the issues surrounding algorithm evaluation.
}

\cventry{Mar 2019}%
    {Data Science Campus Seminar Series}{Office for National Statistics}{}{}{%
        Spoke on my work into evolutionary dataset optimisation. In particular,
        its applications to the field of data simulation and synthesis.
    }

\cventry{Feb 2019}{NHS Wales Modelling Collaborative}{South Wales}{}{}{%
    Invited to speak about how my data-driven approach to my research would
    impact the Cwm Taf Morgannwg Health Board.
}

\cventry{2018--2020}{Advanced Python Workshop}{Cardiff University}{}{}{%
    Founded a group for postgraduate students to engage in monthly,
    tutorial-based sessions about more advanced aspects of Python (such as
    parallelisation and automated testing) followed by a code clinic.
}

\cventry{2018--2019}{STEMLive}{Cardiff University}{}{}{%
    Volunteering, designing and running a stall at the annual outreach event for
    school children in the South Wales region to engage with STEM subjects.
}

\cventry{2018}%
    {EURO Support for NATCOR Bursary}{}{}{}{%
        Financial support to attend NATCOR courses in Approximation Algorithms
        \& Heuristics and Predictive Analysis \& Forecasting.
    }

\section{References available upon request}

\end{document}
