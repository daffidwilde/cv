\documentclass[11pt, sans, a4paper]{moderncv}

\nopagenumbers%
\moderncvstyle{casual}
\moderncvcolor{orange}

\usepackage[default]{Fira Sans}
\usepackage[margin=1.5cm, includefoot, footskip=25pt]{geometry}


\name{Henry}{Wilde\vspace{1pt}}
\address{40 Llanfair Rd}{Cardiff CF11 9QB}{Wales UK}
\email{henrydavidwilde@gmail.com}
\social[twitter]{daffidwilde}
\social[github]{daffidwilde}


\begin{document}

\makecvtitle%

\vspace{-30pt}
\section{PhD thesis (ongoing)}
\cvitem{Title}{%
    \textbf{Utilising machine learning to understand variability in the NHS.}
}
\cvitem{Supervisors}{\small%%
    Dr.\ Jonathan Gillard, Dr.\ Vincent Knight, Mr.\ Kendal Smith (NHS Wales)
}
\cvitem{Description}{\small%
    The purpose of this project is to investigate variation in the service and
    cost of treating hospital patients in the Cwm Taf region of South Wales. My
    thesis focusses on identifying and analysing structures in the data of
    patients, and their treatments and associated clinical processes. By doing
    so, the system can be segmented without an external framework thus more
    clearly illustrating its dependencies and connections. The thesis also
    incorporates an extensive analysis of how algorithms are evaluated, and a
    toolkit on large-scale data science and software development best practices.
}

\section{Education}

\cventry{2017--present}{%
    PhD, Applied Statistics, OR and Data Analytics (expected: Summer 2021)%
}{Cardiff University}{}{}{%
    \begin{itemize}
        \item Research interests: stochastic and healthcare modelling,
            unsupervised learning, game theory.
        \item Key skills: confidence in the principles of data science and
            mathematical programming, ability to develop industrial
            relationships, proficiency in \LaTeX\ and Python.
    \end{itemize}
}

\cventry{2014--2017}{BSc, Mathematics (First Class Honours)}%
    {Cardiff University}{}{}{%
        \begin{itemize}
            \item Key areas of study: mathematical methods for data mining, game
                theory, algorithms and heuristics.
            \item Projects included: building a simulation of a hospital
                emergency department, and a principal analysis of two
                game-theoretic strategies within an Iterated Prisoner's Dilemma
                tournament.
    \end{itemize}
}

\cventry{2012--2014}{A-level qualifications}{Richard Huish College}{Taunton}{}{%
    \begin{itemize}
        \item A* Mathematics, A Further Mathematics, B Physics, A History (AS),
            STEP I
    \end{itemize}
}

\section{Relevant Experience}

\cventry{2019}{Project Allocation}{Cardiff University}{}{}{%
    Two schools within the university have approached me to install a new
    framework for allocating students to projects in their final year. Making
    use of the matching library I maintain, I have been able to reduce the
    bulk workload of the teams down from a week to a matter of seconds. In doing
    so, the provided solution is both stable and student-optimal with almost all
    students the being allocated their first or second choice. This work is
    largely independent and the next step I wish to take is to develop a
    supporting browser-based app for schools to use, and a subsequent
    academic publication.
}

\cventry{2018--present}{Assessment Advisor}{Cardiff University}{}{}{%
    I have acted as part of the assessment team for a Master's module on
    computational methods since the 2017/18 academic year. The assessment
    consists of a two-day hackathon involving several dozen students and my role
    has been to gauge and rank their abilities to work in teams to develop a
    piece of software for a certain mathematical task.
}

\cventry{2018--present}{Maths Support Assistant}{Cardiff University}{}{}{%
    I provide mathematical aid to students from all schools
    within the university as part of the Maths Support Service. This drop-in
    service affords little scope for preparation and requires me to make use of
    my nature as a mathematician \-- to be proactive, analytical and logical \--
    often covering far-reaching branches of mathematics in a session.
}

\cventry{2017--present}{Module Tutor}{Cardiff University}{}{}{%
    Throughout my time as a postgraduate student, I have supported a number of
    modules as a tutor where my primary role is to lead weekly sessions with
    students. These sessions follow up on the content covered in lectures and I
    then assess the progress of the students. This role has given me countless
    opportunities to teach in group and one-on-one settings, and to play a key
    role in several active learning schemes.
}

\clearpage%
\section{Publications and projects}

\cventry{2019}{Evolutionary Dataset Optimisation: learning algorithm quality
    through evolution}{}{}{}{%
        \begin{itemize}
            \item e-Print: \url{https://arxiv.org/abs/1907.13508}
            \item Accompanying Python library:
                \url{https://github.com/daffidwilde/edo}
        \end{itemize}
    }

\cventry{2018}{Matching: a package for solving matching games}{}{}{}{%
    \begin{itemize}
        \item Python library: \url{https://github.com/daffidwilde/matching}
        \item Documentation: \url{https://readthedocs.io/matching}
    \end{itemize}
}

\section{Additional Activities}

\cventry{May 2019}{Welsh Mathematics Colloquium}{Gregynog Hall}{}{}{%
    Gave an in-depth talk on the mathematical principles of evolutionary dataset
    optimisation and some of the issues surrounding algorithm evaluation.
}

\cventry{Mar 2019}{Data Science Campus Seminar Series}%
    {Office for National Statistics}{}{}{%
        Spoke on the application of my work into evolutionary dataset
        optimisation. In particular, its applications to the field of data
        simulation and synthesis.
    }

\cventry{Feb 2019}{NHS Wales Modelling Collaborative}{South Wales}{}{}{%
    Invited to speak about how I have approached my work for the Cwm Taf
    University Health Board. This involved an introduction to clustering and
    discussion around the pitfalls of not allowing data to drive decision
    making.
}

\cventry{2018--present}{Advanced Python Workshop}{}{}{}{%
    Founded a group for postgraduate students to engage in tutorial-based
    sessions about more advanced aspects of Python (such as parallelisation and
    automated testing) followed by a code clinic.
}

\cventry{2018}{%
    \href{https://www.euro-online.org/}{EURO} Support for NATCOR Bursary%
    }{}{}{}{%
        Financial support to attend NATCOR courses in Approximation Algorithms
        \& Heuristics and Predictive Analysis \& Forecasting.
    }

\cventry{2017--present}{%
    \href{http://www.pydiff.wales/}{PyDiff}: a Cardiff-based, Python discussion
    group%
}{}{}{}{%
        I have taken on responsibilities in
        the organisation of the group as well as speaking about the usefulness
        of specific libraries for handling larger-than-memory datasets with
        Python.
}

\cventry{2016--present}%
    {\href{http://www.theorsociety.com/Pages/Regional/swords.aspx}{%
        South Wales Operational Research Discussion Society (SWORDS)%
    }}%
    {}{}{}{%
        I have been an active member, regularly attending meetings and talks
        since becoming a member of the OR Society during my undergraduate
        degree.
    }

\section{References}

\begin{cvcolumns}
    \cvcolumn{Academic}{%
        \begin{tabular}{l}
            \tiny\\
            Dr.\ Vincent Knight\\
            Senior Lecturer\\
            School of Mathematics, Cardiff University\\
            \texttt{knightva@cardiff.ac.uk}
        \end{tabular}
    }
    \cvcolumn{Personal}{%
        \begin{tabular}{l}
            \\
            Mr.\ Alex Beeston\\
            Manager and Owner\\
            The High Cross, Tottenham\\
            \texttt{alexbeeston1988@gmail.com}
        \end{tabular}
    }
\end{cvcolumns}

\end{document}
